%!TEX root = index.tex
\chapter{Conclusion}

Ce rapport a présente mes expériences professionnelles dans l'établissement Asert Serviços e Tecnologia da Informação, une petite société informatique qui propose des solutions de systèmes de gestion intégrés à d'autres entreprises, principalement à celles d'aide médicale. 

L'opportunité de réaliser un stage ingénieur chez une entreprise de développement logiciel a été très enrichissante, et ces huit semaines de travail ont été très productives. Ce stage m'a donné une grande ouverture d'esprit en voyant une équipe très capacité et m'a motivé dans un cadre académique pour produire des technologies appliquées à des problèmes réels du monde corporatif.

L'environnement amical et stimulant favorise un engagement et un approfondissement interdisciplinaire remarquable. Le fait d'être un petit groupe de développement fait qu'il soit nécessaire de croiser de différentes connaissances et penser aux problèmes à partir d'angles inhabituels. Dans ce sens, le bureau crée une atmosphère qui invite les personnes qui travaillent dans des domaines distincts à interagir et se communiquer même dans des circonstances non-professionnelles. 

Par rapport aux résultats de mon travail, j'ai pu constater une amélioration de ce que je développais, non pas seulement en question de vitesse de travail mais aussi de qualité de code. À la fin de mon séjour, le développeur responsable ne faisait plus de vérifications, car il était à moi de le faire de façon rigoureuse et progressive. 

Le poste que j'ai occupé était sans doute plus proche de celui d'un ingénieur que dans le stage ouvrier à la première année : il y avait des objectifs clairs à remplir, des spécifications, des délais, et des compétences techniques requises. Tout de même, au fur et a mesure que je concevais des fenêtres similaires les unes des autres, j'avais l'impression de ne pas créer de nouvelles fonctionnalités, mais tout simplement de rester sur la méthode de travail ; c'était aux développeurs et à l'architecte d'améliorer le framework de l'entreprise, tandis que j'en étais seulement un utilisateur. J'ai conclu finalement que ce sentiment est normal dans le cadre d'un stage d'apprentissage, vu que j'ai eu très peu de contact avec les bases fondatrices du système, mais cela m'a donne l'envie d'assumer l'un de ces postes dans l'avenir.

La continuation du travail que j'ai effectué a déjà été planifiée et sera finie avant la fin de l'année. L'objectif principal est d'avoir un livrable préliminaire au plus vite possible pour que le client puisse faire des suggestions et l'équipe de développement reformule le système. 