\chapter{Déroulement de la Mission}



historico
-criacao
-numero pessoas
-projetos
-escritorio
-estrutura
-mercado
-inovacao

1. evolute -> varios projetos, dont planos de saude
	marco aurelio (diretor administrativo) + roberto (analista de sistemas) saem para tocar a Asert, do grupo evolute mas especifica em saude
	de 2 anos pra ca eles sairam definitivamente da evolute
2. hoje 12 pessoas - microempresa -. em 2010, 200 pessoas por causa de contrato com o estado: medicos, auditores, etc. contrato acabou, ficou apenas a equipe principal
3. tres salas no empire center, ~110m^2. no inicio, 1 sala
4. 2 seotres : comercial (vendedores independentes, que trabalham fora da empresa, fazendo o contato com clientes e em busca de novos clientes)-financeiro(Marco = gestor) e desenvolvimento de TI (Aladin : gestor, analista de negocios)
5. Projetos
	projetos EVOLUTE -> CCRI -> Ipasgo -> Palmas/TO 
	Ministerio Publico/GO, celgmed, Emater http://www.emater.go.gov.br/
	centro de pesquisa de alimentos (CPA)-UFG, hoje em fase de sustentacao/assistencia
	gestao de sanitaria-frigorificos (Friboi, Minervas): linha de producao, bacteria na carne, redes Bayesianas - sistema especialista
6. Asert ciclo : gestao de reciclagens
7. concorrencia
	20 anos atras : sistemas personalizados
	pouco tempo atrás: boom dos ERPs, fez diminuir bastante o preco de todos sitemas de gestao. todavia, dificil/cara adaptacao selon les caracteristiques du client. chaque entreprise tem a sua cara
	hoje: retorno devagar a sistemas personalizados
	.. concorrencia: 
		erps de comercio; empresas como Canyon Software, Mega Assistencia, Interage Tecnologia, Implanta TI. Todas sao pequenas e medias emrpesas.
		BPO e BPM. http://processplatsen.ibissoft.se/?q=en/node/40
		empresas de SaaS, como a Apta
		ha apenas uma concorrencia direta, do Paraná, que também é uma empresa especializada em softwares de gestao de planos de saude, mas eles nao tem uma grande implantacao em Goiás.
8. mercado
	mercado de TI no brasil: 99\% das empresas brasileiras sao MPE, micro e pequenas empresas, com poucos funcionarios e, dessas, 90\% fecham em 5 anos. Dificuldades tributarias, burocraticas, de implantacao no mercado, etc.
	buscando novos clientes (antigos usuarios de ERP que tao procurando algo mais personalizado). Esse systeme expert da friboi nao existe em lugar nenhum do mundo (por ser bastante especifico) e é um novo produto: é uma das maiores apostas da Asert no momento.
9. inovacao
	parceria com a instituicao governamental FAPEC (com a ideia do projeto) + instituicao de ensino UFG (com os modelos teoricos, systeme expert) no projeto da friboi (desenvolvido pela expertise tecnologica da Asert). se der certo, todos ganham participacao
