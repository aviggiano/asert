\chapter{Deroulement de la mission}

La mission qui m'a été affecte a été celle d'aider dans le développement logiciel d'un progiciel de gestion intégré pour le service d'aide médicale CELGMED. 

Au cours des deux premières semaines, l'architecte logiciel responsable par le projet m'a formé dans le framework Asert, c'est-a-dire, des fonctions et librairies qui ont été développées au sein de l'entreprise spécifiques a la mise en œuvre des projets.

Encore dans le premier mois de mon stage, des notions d'architecture logicielle m'ont été présentées, en particulier celle des trois couches et la MVC (Model View Controller), qui seront détaillées par la suite.

Après cette période de formation, j'ai pu participer au développement du système \gls{ERP}. J'étais charge de l'exécution de plusieurs fenêtres, dès la conception de l'interface graphique jusqu'à la création des entités dans la base de donnes et des classes dans le programme.

Le cycle de développement logiciel n'a pas été un thème de grande importance dans ma mission, vu que l'équipe de production était très limitée et le projet était déjà en phase de codage. Tout de même, le chef de projet m'a expliqué les étapes du processus habituel de l'entreprise, présentées dans les pages qui suivent.

Toutes ces étapes de la mission seront décrites avec plus de détails dans ce chapitre.  

% 1o mes -> formacao no framework
%% falar bastante do framework
% nocoes de arquitetura de software
%% falar bastante das arquiteturas
% execucao pratica e programacao de telas
%% falar que nao tem muitas figuras devido a contraintes de confidentialite



% flex --> alternative to jsf, jsp