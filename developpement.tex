%!TEX root = index.tex
\section{Développement}

% intro
% como era o time e do que cada pessoa era responsavel
% passo a passo da minha routine de dev
	% o que os outros faziam, o que eu fazia
% exemplo prático na criacao de alguma tela (das que geram relatorio).

Le deuxième mois de mon stage a été concentré sur le développement du système de gestion d'adhérents de CELGMED. J'étais charge de la conception et création de plusieurs entités Java et fenêtres graphiques Flex, et par la liaison avec les bases de données SQL Server déjà existantes.

L'équipe informatique était composé du chef de projets, d'un architecte logiciel, d'un développeur base de données, d'une analyste de systèmes, d'un développeur logiciel et de deux stagiaires, moi y compris. Du fait de n'être pas une équipe nombreuse, les tâches étaient souvent confondues et le développement était très agile : l'architecte logiciel touchait aussi au code, le développeur logiciel aidait le responsable des bases de données, et les stagiaires les aidait dans toutes les aires de la programmation.

De manière générale, 

