\documentclass[12pt]{article}

% Preambulo
\usepackage[frenchb]{babel}
\usepackage{float}
\usepackage[utf8]{inputenc}  
\usepackage{lmodern}
\usepackage[T1]{fontenc}  
\usepackage{graphicx}
\usepackage{indentfirst}
\usepackage{courier}
\usepackage{geometry}
\usepackage{glossaries}
\usepackage{eurosym}
\newcommand {\cod} [1] { \texttt {#1}} 
\newcommand {\p} [1] { \paragraph {#1}}
\usepackage{hyperref}
\hypersetup{
    colorlinks,
    citecolor=black,
    filecolor=black,
    linkcolor=black,
    urlcolor=black
}
\usepackage[urldate=comp,dateabbrev=false,sorting=none]{biblatex}
\bibliography{bibliography}

\title{Rapport de Stage 2A}
\author{Antônio Guilherme \textsc{Ferreira Viggiano}}
%\date{}
\makeindex

\geometry{paperwidth=210mm,paperheight=297mm,
top=30mm,bottom=30mm,
left=30mm,right=30mm}
% Documento

\makeglossaries

\begin {document}


\begin{titlepage}
\begin{center}
% Upper part of the page

\begin{figure}[ht]
\begin{minipage}[b]{0.5\linewidth}
\centering
\includegraphics[width=0.7\textwidth]{ecm}
\label{logo_ecm}
\end{minipage}
\hspace{0.5cm}
\begin{minipage}[b]{0.5\linewidth}
\centering
\includegraphics[width=0.9\textwidth]{logo_asert}
\label{logo_asert}
\end{minipage}
\end{figure}

%\textsc{\LARGE Intelligence Artificielle et Prolog}\\[1.5cm]


% Title
\vspace{2cm}
\hrule  ~\\[0.4cm]
{ \huge \bfseries Rapport de stage 2a}\\[0.4cm]

\hrule  ~\\[1.5cm]
\large Du xx au yy

% Author and supervisor
\vspace{4cm}
\large\emph{Elève :} Bruno \textsc{Brizida Dreux} ~\\
\vspace{0.5cm}
\large\emph{Tuteur pédagogique :} M. Serge \textsc{Xella}

\large\emph{Tuteur de stage :} M. Laurent \textsc{Herisse}




\vfill

% Bottom of the page
{\large\today}

\end{center}

\end{titlepage}


%Glossary
\renewcommand*{\glspostdescription}{. Page:}
\renewcommand*{\glossaryname}{}

\pagebreak	
\tableofcontents
\input{glossaire.tex}

\pagebreak
\section{Remerciements}

\chapter*{Remerciements}

Je voudrais remercier à tous ceux qui ont contribué pour le bon déroulement de ce stage, en particulier à :

\vspace{1em}

\begin{description}
\item[Mme Mirta JUAREZ et Mme Cécile LOUBET --] pour les réponses rapides à toute question concernant le stage à l'international.
\item[M. Marco Aurélio RABELO --] président-directeur général de Asert, pour être si proche des employés.
\item[M. Clemenceau Roberto DA SILVA --] directeur de technologie de l'entreprise, qui m'a bien accueilli par email et personnellement au sein de l'entreprise.
\item[M. Thiago RODRIGUES PEREIRA --] architect logiciel, pour m'avoir formé dans le framework de l'entreprise et m'avoir introduit aux bonnes pratiques en génie logiciel.
\end{description}


\pagebreak
\section{Glossaire}
\printglossary

\pagebreak
%!TEX root = index.tex
\chapter{Bibliographie}

\cite{uol-mercado-tic,
pib-brasil
}

\printbibliography[heading=bibintoc]


\end{document}
