%!TEX root = index.tex
\chapter{Introduction}

L'École Centrale Marseille propose dans son cursus aux élèves de découvrirent le monde économique et ses contraintes et de bien appréhender la complexité du métier d'ingénieur, au travers d'un stage en entreprise d'une durée de 8 semaines au minimum. Celle-ci est une opportunité extrêmement enrichissante dans plusieurs aspects, complémentaire au stage ouvrier de la première année. 

Ce rapport présente les expériences dans l'établissement Asert Serviços e Tecnologia da Informação, une petite société informatique qui propose des solutions de systèmes de gestion intégrés à d'autres entreprises, principalement à celles d'aide médicale.

Le déroulement de ce stage a été du 3 juin 2013 au 26 juillet 2013, avec une durée totale de huit semaines, où j'ai occupé le poste de développeur logiciel dans le Centre de Développement de Systèmes à Asert. 

Ce document a pour objectif de montrer de façon détaillée les expériences vécues lors du stage, et pour cella il est divisé en trois parties :  la présentation de l'entreprise, la description du poste occupé, avec une mise en évidence du processus de développement logiciel, du cycle de vie du produit et de son architecture, et la conclusion contenant les remarques personnelles de l'élève en ce qui concerne cette expérience.
