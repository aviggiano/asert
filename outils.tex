%!TEX root = index.tex
\section{Outils de développement}

Les solutions Asert sont presque toujours supportées par les mêmes outils de développement logiciel, dans le modèle suivant :
\begin{description}
\item[Microsoft Windows] -- le système d'exploitation utilisé aussi bien pour le développement que pour l'hébergement des applications.
\item[Adobe Flash Builder] -- aussi connu comme Adobe Flex \cite{flex}, l'environnement de développement (IDE) construit en tant qu'une couche sur la plate-forme \Gls{eclipse} destinée au développement d'applications internet pour la plate-forme Adobe Flash.
\item[GlassFish] -- un serveur d'applications en stage final de développement qui est supporté par Oracle et la communauté GlassFish en utilisant une licence open source. GlassFish Server \cite{glassfish} est généralement publié avec le soutien de la dernière plate-forme \cite{jee}, bien en avance sur les autres implémentations de serveurs d'applications.
\item[Microsoft SQL Server] -- système de gestion de base de données relationnelles qui stocke et de récupère les données demandées par d'autres applications logicielles, que ce soit ceux sur le même ordinateur ou celles en cours d'exécution sur un autre ordinateur dans un réseau, comme l'Internet \cite{sql-server}.
\item[Hibernate] -- framework open source gérant la persistance des objets Java en base de données relationnelle. Hibernate apporte une solution aux problèmes d'adaptation entre le paradigme objet et les \gls{SGBD} en remplaçant les accès à la base de données par des appels à des méthodes objet de haut niveau \cite{hibernate}.
\item[Spring] -- framework libre qui facilite la construction et définition de l'infrastructure d'une application Java, ainsi comme des tests de routine \cite{spring}. Il rend possible l'inversion de contrôle, un patron d'architecture qui fonctionne selon le principe que le flot d'exécution d'un logiciel n'est plus sous le contrôle direct de l'application elle-même (et donc du programmeur qui l'a développé) mais du framework ou de la couche logicielle sous-jacente \cite{wiki-ioc}.
\item[BlazeDS]
\item[iReport Designer]
\end{description}

% fkex --> alternative to jsf, jsp