\chapter{Présentation de l’entreprise}
\section{Historique}

Basée dans la capitale de Goiás, Asert a été fondée en 2002 par deux employés de Evoluti Tecnologia e Serviços (Evoluti Technologie et Services) avec l'idée de fournir des systèmes de gestion au secteur de la santé, un domaine qui ne possédait pas les compétences techniques nécessaires à l'intégration de tous ses services. Tout de même, Asert n'a éte qu'une branche rattachée à Evoluti jusqu'en 2008, lorsque l'entreprise a réussi à avoir suffisamment de projets indépendants.

De 2008 à 2010, la société a travaillé fondamentalement avec des systèmes pour des mutuelles de santé. Récemment, suivant les tendances du marché, l'organisation a élargi son champ d'application jusqu'à l'externalisation de tout type de service spécialisé.

\section{Implantation et Taille}

Asert est une micro-entreprise localisée dans la région commerciale de Goiânia, dans un bureau d'approximativement 110 m$^2$, actuellement avec un effectif de 12 employés. En 2010, grâce à un projet en partenariat avec l'état de Goiás, l'entreprise comptait 200 personnes, y compris des médecins, des auditeurs et consultants, des professionnels de TI, etc. Cependant, à la fin de l'engagement, Asert a dû mettre à terme les contrats de travail à durée déterminée et rester avec son équipe principale.

\section{Type de structure}

L'entreprise se partage en deux secteurs, le Centre de Développement de Systèmes, où quatre développeurs sont gérés par un chef d'équipe, et le Centre Commercial-Financier, où des commerciaux, compteurs et administrateurs sont dirigés par le directeur financier.

L'équipe commerciale se charge de trouver de nouveaux clients et de faire une analyse préalable de la situation de l'entreprise. Le chef de projets du centre de développement fait ensuite l'analyse technique du système de gestion souhaité, en prenant note de toutes les particularités spécifiques aux règles métier (ou ```business rules'' en anglais) \cite{regle-metier} de l'application. Une fois le besoin du client complètement identifié, l'équipe de développement décide la structure ``physique'' du système -- quelle base de données relationnelle utiliser, quelle langage de programmation ou technologie la plus adaptée au problème, etc. -- et passe à l'étape de codage.

L'équipe financière travaille en parallèle de toutes ces étapes, en analysant les coûts de travail et d'équipements.

\section{Gamme de produits}

Même si Asert est spécialisée en services de santé, le groupe travaille  sur deux grandes gamme de projets, ceux de technologies et de l'information et ceux de systèmes intégrés. 

\subsection{Technologies de l'information}

Dans l'axe des sciences de l'information et de la communication, l'entreprise propose des services de audit et conseil, de développement de systèmes, d'informatique décisionnelle (en anglais \textit{Business Intelligence}) et de gestion.

\subsubsection{Audit et conseil}

Le rôle du conseil en technologies de l'information est de fournir au client un bilan de la situation actuelle de l'entreprise, d'identifier les problèmes liés à la politique, à la structure, aux procédures et aux méthodes, afin de recommander et d'aider à mettre en \oe{}uvre les mesures appropriées aux étapes d'innovation et de croissance de l'entreprise. Après ce diagnostic fondé sur le besoin du client et les résultats attendus, Asert s'occupe du développement de solutions spécialisées et de projets techniques, avec des professionnels qualifiés au travail. Le service de \textit{consulting} comprend :

\begin{itemize}
\item Mise à niveau et modernisation de la technologie de la société;
\item Évaluation, sélection et embauche de services et logiciels tiers;
\item Préparation de projets de logiciels;
\item Gestion de projets.
\end{itemize}

\subsubsection{Développement de systèmes}

Ce n'est pas toujours que les systèmes de gestion disponibles sur le marché répondent aux demandes des organisations. C'est en vue de cela que Asert propose des systèmes sur mesure, en produisant des outils personnalisés en fonction des besoins des clients, de manière à assurer une plus grande productivité et amélioration de la gestion internet des entreprises.

Asert utilise les plus modernes méthodologies de développement logiciel, utilisant des méthodes de contrôle de version et de testes unitaires afin d'avoir un rendu selon les exigences et les besoins des clients.

\subsubsection{Informatique décisionnelle -- \textit{Business Intelligence}}

Les systèmes d'information sont responsables par la manipulation d'une quantité de données très importante au sein des entreprises. L'informatique décisionnelle, aussi connue par la traduction anglaise ``intelligence d'affaires'', est le processus de collecte, d'organisation, d'analyse, de partage et de suivi de ces données, afin d'extraire des informations et des indicateurs pour soutenir la gestion de l'établissement. Les services B.I. de Asert incluent :

\begin{itemize}
\item Analyse et conception d'environnements de gestion de l'information;
\item Analyse de la qualité des données;
\item Gestion des données;
\item Rendu de rapports dynamiques destinés aux gestionnaires de l'entreprise.
\end{itemize}

\subsubsection{Gestion des T.I.}

L'équipe de Asert est axée sur la satisfaction du client, prête à résoudre les difficultés dans l'accès ou l'utilisation des technologies de l'information. Elle fournit également aux utilisateurs un centre d'assistance pour résoudre à tout problème technique lié à ses services, afin de rétablir le fonctionnement normal des activités dès que possible, minimisant ainsi les impacts commerciaux causés par des pannes informatiques.

La gestion des technologies de l'information apporte plusieurs avantages aux organisations, comme la construction d'un lien entre les T.I. et la gestion d'entreprise, la réduction des coûts, l'acheminement des appels aux équipes spécialisées, le soutien aux utilisateurs finaux  et l'amélioration de la qualité des services.

\subsection{Systèmes Intégrés}

Les systèmes intégrés de l'entreprise se partagent en deux secteurs : la médecine, celle-ci divisé en Asert Santé et Asert Méd, et l'industrie, avec Asert Cycle.

\subsubsection{Asert Santé}

Asert Santé est le type de système informatisé pour la gestion des sociétés d'assistance médicale. Développé avec les technologies les plus modernes, le système se traduit par un portail web comme interface d'administration d'utilisateurs et d'adhérents, et de bases de données relationnelles qui manipulent toutes les informations. Ces systèmes sont en conformité avec les spécifications de l'Agence Nationale de Santé brésilienne.

Ces caractéristiques techniques, alliées à une interface conviviale et intuitive, rend le système facile à utiliser, sans demander beaucoup de ressources matérielles. Techniquement, le système est divisé en modules indépendants, qui sont intégrés selon la réalité et l'organisation de l'entreprise de santé. Ce type de développement ``en couches'' sera détaillé par la suite. 

Les particularités de Asert Santé sont :

\begin{itemize}
\item Modules intégrés;
\item En accord avec les normes de l'Agence Nationale de Santé brésilienne;
\item Rapidité, efficacité et fiabilité dans les tâches effectuées;
\item Vérification électronique;
\item Normalisation des tables de facturation et des rapports de gestion;
\item Facturation électronique.
\end{itemize}

\subsubsection{Asert Méd}

Asert Méd est un outil idéal pour la gestion des cliniques, des hôpitaux, des laboratoires et des bureaux de médecin. Ce système vise à faciliter le service clients, la facturation et la présentation des comptes, adapté aux besoins du client.

Caractéristiques et avantages:

\begin{itemize}
\item Modules intégrés;
\item En accord avec les normes de l'Agence Nationale de Santé brésilienne;
\item Enregistrement unique des patients;
\item Assistance aux patients affiliés ou adhérents à une mutuelle de santé;
\item Émission de demandes de visites médicales dans les modèles des normatifs;
\item Contrôle des rapports médicaux personnalisés et de factures de soins médicaux et hospitaliers;
\item Contrôle des stocks multiple (entrepôt, pharmacie, soins infirmiers, etc.);
\item Gestion financière;
\item Graphiques et rapports de gestion;
\item Dossier médical électronique d'un patient.
\end{itemize}

\subsubsection{Asert Cycle}

Asert Cycle est un système développé avec les dernières technologies sur le marché pour la gestion d'industries, permettant de façon innovante le contrôle, le suivi et l'intégration des politiques de management, de production, de commerce et fiscales des entreprises. Ce système se caractérise par :

\begin{itemize}
\item Gestion de stock;
\item Suivi du produit dans toutes les étapes de fabrication;
\item Expédition et vente des produits;
\item Émission de la facture;
\item Gestion de la production.
\end{itemize}

\section{Marché}
\subsection{Le marché des technologies de l'information et communication au Brésil}


8. mercado
	mercado de TI no brasil: 99\% das empresas brasileiras sao MPE, micro e pequenas empresas, com poucos funcionarios e, dessas, 90\% fecham em 5 anos. Dificuldades tributarias, burocraticas, de implantacao no mercado, etc.


\subsection{La position de Asert dans le marché}

Dans les années 1990 et 2000, le boom des des progiciels de gestion intégré (ou en anglais ERP,  ``Enterprise Resource Planning'') a encouragé la création de plusieurs entreprises de de logiciels en tant que service (de l'anglais ``Software as a Service'', SaaS) dans l'état de Goiás, tels que Canion Software, Interagi Tecnologia, et Apta. Cependant, ce phénomène de haute concurrence a fait tomber les prix des ERPs et a empêché la croissance des petites et moyennes entreprises de technologie.

Ce que l'on constate aujourd'hui est le retour aux systèmes personnalisées. Les clients sont plus exigeantes dans les systèmes adaptés et ne veulent plus de logiciels génériques, souvent critiqués d'être de difficile adaptation et de ne pas représenter l'identité de l'entreprise. Avec ce changement de valeurs du marché, Asert a gagné de nouveaux clients et a assuré sa position en tant qu'entreprise de développement de solutions individualisées.

%..	buscando novos clientes (antigos usuarios de ERP que tao procurando algo mais personalizado). Esse systeme expert da friboi nao existe em lugar nenhum do mundo (por ser bastante especifico) e � um novo produto: � uma das maiores apostas da Asert no momento.

%pouco tempo atr�s: boom dos ERPs, fez diminuir bastante o preco de todos sitemas de gestao. todavia, dificil/cara adaptacao selon les caracteristiques du client. chaque entreprise tem a sua cara
%hoje: retorno devagar a sistemas personalizados

%BPO e BPM. http://processplatsen.ibissoft.se/?q=en/node/40

%ha apenas uma concorrencia direta, do Paran�, que tamb�m � uma empresa especializada em softwares de gestao de planos de saude, mas eles nao tem uma grande implantacao em Goi�s.

\section{La situation de l'entreprise}

Au cours du temps, les projets de Asert ont évolué non pas seulement en taille mais aussi en type. Lorsque l'entreprise était rattaché au groupe Evoluti. Le premier projet de Asert a été la mise en ouvre d'un système de gestion pour l'association d'assistance médicale ``Ipasgo'' (Institut de l'assistance publique de l'Etat de Goias). Ensuite, le système s'est adapté à la ville de Palmas, capitale de l'état de Tocantins. Aujourd'hui ces deux projets sont en phase d'assistance -- l'étape finale de toutes les solutions d'externalisation de l'entreprise.

Actuellement, l'entreprise subit des transformations structurelles afin de pouvoir agir comme un moteur d'innovation de l'état. En partenariat avec l'instituition gouvernementalle FAPEC et avec l'instituition d'enseignement et de recherche Université de Goias, Asert est en train de developper un système expert pour la gestion sanitaire de rèfrigèrateurs pour le client Friboi. Toutes les ètapes de production, ainsi comme la qualitè de la viande bovine seront analisèes avec des rèseaux bayèsiens afin de minimiser les pertes par contamination bacterienne. Ce projet est encore en cours de conception et n'a pas de date prèvue de livraison.

%..Projetos EVOLUTE -> CCRI -> Ipasgo -> Palmas/TO 
%	Ministerio Publico/GO, celgmed, Emater http://www.emater.go.gov.br/
%	centro de pesquisa de alimentos (CPA)-UFG, hoje em fase de sustentacao/assistencia
%	gestao de sanitaria-frigorificos (Friboi, Minervas): linha de producao, bacteria na carne, redes Bayesianas - sistema especialista
%6. Asert ciclo : gestao de reciclagens


%..9. inovacao
%	parceria com a instituicao governamental FAPEC (com a ideia do projeto) + instituicao de ensino UFG (com os modelos teoricos, systeme expert)% no projeto da friboi (desenvolvido pela expertise tecnologica da Asert). se der certo, todos ganham participacao


